\documentclass[11pt]{article}

\usepackage{amsmath, amssymb}
\usepackage{graphicx}
\usepackage{booktabs}
\usepackage{geometry}
\usepackage{hyperref}

\geometry{margin=1in}

\title{Dynamic Cross-Asset Correlation Structure and Systemic Risk}
\author{Your Name}
\date{\today}

\begin{document}
\maketitle

\begin{abstract}
We study the time-varying correlation structure across equities, foreign exchange, rates, and commodities.
Using rolling correlation matrices, heatmap visualization, and spectral decomposition via Principal Component
Analysis (PCA), we show that diversification is highly regime-dependent and collapses during periods of market stress.
The first principal component of the correlation matrix acts as a robust proxy for systemic risk.
We further document a structural shift in cross-asset correlations following the COVID-19 crisis.
\end{abstract}

\section{Introduction}
Portfolio diversification relies on stable correlation assumptions.
Empirically, correlations are non-stationary and rise sharply during market stress,
precisely when diversification is most needed.
This paper analyzes correlation dynamics at the portfolio level
and quantifies systemic risk using the evolving dependence structure.

\section{Related Literature}
Our work relates to the literature on correlation breakdowns during crises
(Longin and Solnik, 2001), systemic risk measurement (Kritzman et al.),
and dynamic dependence modeling (Engle, 2002).

\section{Data}
We analyze daily returns for equities, FX, rates, and commodities from 2015 onward.
All prices are adjusted close and synchronized on a common trading calendar.

\section{Methodology}

\subsection{Rolling Correlation Estimation}
Let $R_t \in \mathbb{R}^N$ denote asset returns.
For a rolling window $W$, the correlation matrix is:
\[
C_t = \text{Corr}(R_{t-W:t})
\]

\subsection{Average Correlation Level}
We define the average pairwise correlation:
\[
\bar{\rho}_t = \frac{2}{N(N-1)} \sum_{i<j} \rho_{ij,t}
\]
This serves as a proxy for global correlation regimes.

\subsection{Spectral Decomposition}
Each correlation matrix is decomposed as:
\[
C_t = V_t \Lambda_t V_t^\top
\]
The proportion of variance explained by the first eigenvalue:
\[
\text{PC1}_t = \frac{\lambda_{1,t}}{\sum_i \lambda_{i,t}}
\]
is interpreted as a measure of systemic risk concentration.

\section{Empirical Results}

\subsection{Rolling Heatmaps}
Figure~\ref{fig:heatmap} shows rolling correlation heatmaps.
Correlation clustering intensifies during stress periods.

\subsection{Systemic Risk Indicator}
Figure~\ref{fig:pca} plots the PC1 time series.
Sharp spikes coincide with known crisis periods, including COVID-19.

\subsection{Pre- vs Post-COVID Analysis}
Post-COVID correlations exhibit persistently higher baseline levels,
suggesting reduced long-term diversification benefits.

\section{Discussion}
Correlation concentration reflects market-wide risk transmission
and highlights the fragility of static diversification assumptions.

\section{Limitations}
The analysis relies on linear correlations and PCA.
Nonlinear dependence and tail risk are not explicitly modeled.

\section{Conclusion}
Dynamic correlation modeling provides critical insights into systemic risk.
Portfolio construction frameworks must account for regime-dependent dependence structures.

\bibliographystyle{plain}
\bibliography{references}

\end{document}
